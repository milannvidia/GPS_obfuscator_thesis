%aanleiding
\section*{Inleiding}
\subsection*{aanleiding}
Veel moderne applicaties verzamelen de locatiegegevens van hun gebruikers, deze locatiegegevens kunnen dan gebruikt worden voor crowd-insights dat nuttig is voor verscheidene onderzoeken, zoals mobiliteitsstudies en stadsplanning. Echter, de hoge nauwkeurigheid en frequentie van deze gegevens brengen aanzienlijke privacyrisico’s met zich mee. In verkeerde handen kunnen locatiegegevens leiden tot sociale profilering of zelfs spionage, wat de persoonlijke veiligheid van gebruikers in gevaar kan brengen.
\subsection*{afbakening}
%afbakening
Het doel van deze thesis is om een data stream obfuscator te implementeren als een Android-bibliotheek. Deze obfuscator voegt ruis toe aan zowel de locatie- als tijdgegevens, zodat individuele bewegingspatronen niet meer naar specifieke personen kunnen worden herleid. Hiermee wordt het risico op het lekken van gevoelige informatie verminderd. Daarnaast worden de prestaties van deze obfuscator geëvalueerd en vergeleken met de "Approximate location"-functie die sinds Android 12 beschikbaar is. Op deze manier kan de effectiviteit van bestaande en nieuwe privacymaatregelen voor mobiele apparaten worden onderzocht.
\subsection*{Huidige kennis}
%Huidige kennis
\subsection*{theoretische en praktische relevantie:}
%theoretische en praktische relevantie:
De ontwikkeling van deze bibliotheek zou de obfuscatie berekening van de server naar de client verhuizen. Waardoor gevoelige informatie (bv. een privacy bubbel) lokaal op het apparaat blijven en enkel geobfusceerde locaties worden doorgestuurd en zo data lekken minder schadelijk zijn.
\subsection*{eventueel info over android controleren?}
%eventueel info over android controleren?