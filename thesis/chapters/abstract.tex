%waarom?
%   probleem
Veel applicaties verzamelen de locatiegegevens van gebruikers, deze locatiegegevens kunnen dan gebruikt worden voor crowd-insights dat nuttig is voor verscheidene onderzoeken. Echter vormt de combinatie van de hoge nauwkeurigheid en frequentie van deze gegevens een potentieel privacyprobleem, waarbij sociale profilering of zelfs spionage mogelijk is als deze informatie in verkeerde handen komt. 
%   doelstelling en onderzoeksvraag
%hoe? methoden -> 4zinnen gemengd met doestelling?
Het doel van deze thesis is om een data stream obfuscator te implementeren als een Android-bibliotheek. Deze obfuscator plaatst noise op zowel de locatie als op de tijd waardoor individuele traces niet meer kunnen terug lijden naar één individu, zo kan eventuele gevoelige informatie niet lekken. Verder worden de prestaties van deze obfuscator dan vergeleken met de recente "Approximate location"-implementatie in Android 12. Hiermee kan de vooruitgang in privacymaatregelen op mobiele telefoons worden onderzocht.
%resultaten enkel ingaan op hoofdvraag
%de cunlusie is daarom ...

The thesis should contain an abstract in Dutch \emph{and} in English.
You should aim for anywhere between 500 and 1500 words (per abstract).

\underline{Trefwoorden:} latex, thesis, stijl